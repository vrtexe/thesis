\newpage
\pagenumbering{roman}
\setcounter{page}{2}

\renewcommand{\abstractname}{\large Апстракт}
\phantomsection
\addcontentsline{toc}{section}{\abstractname}


\begin{abstract}

    Корисниците се соочуваат со големи предизвици при хостирање на апликации, бидејќи покрај технологиите потребни за развој на самата апликација, неопходно е да поседуваат и познавање за околината во која таа се хостира, без разлика дали станува збор за инфраструктура во облак или за посветен сервер, како и познавање на платформата на која се извршува. Со цел да се надминат овие предизвици, изградено е решение за хостирање на апликации во облак кое овозможува лесно управување, распоредување и увид во состојбата на апликациите преку едноставен и интуитивен кориснички интерфејс. 
    
    За реализација на ова решение беа искористени современи технологии како Kubernetes, Spring Framework и други придружни алатки, кои овозможуваат интеграција на повеќе сервиси за управување и распоредување на корисничките апликации. Главната цел на архитектурата на решението е обезбедување на поголема едноставност за крајните корисници, при што истовремено се задржува флексибилноста и можноста за понатамошно проширување на системот. Овие аспекти се оставени во доменот на администраторите на платформата.
    
    Покрај едноставноста, решението овозможува и увид во состојбата на апликациите во реално време, поддршка за градење на контејнерски слики, како и лесно распоредување на апликациите во рамките на користената платформа. Додека за администраторите се дозволува додавање на можности за корисниците преку користење на шаблони за градење на сликите, како и управување со ресурсите на платформата. 
    
    \textbf{Клучни зборови:} \textit{Хостирање, Kubernetes, градење на слики, Docker, конинуирана интеграција, респоредување на апликации}
\end{abstract}
