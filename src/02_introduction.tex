\newpage
\pagenumbering{arabic}
\setcounter{page}{1}


\section{Вовед}

Хостирање на апликации во облак е современ пристап кој овозможува флексибилност, скалабилност и лесно управување со апликациите, каде што сите опции за конфигурација од вмрежување до апликациски конфигурации се овозможени преку кориснички интерфејс, кој најчесто знае да биде доста комплициран за корисниците кои немаат претходно искуство со ваков тип на платформи. Целта на ова решение е да се овозможи платформа која ќе овозможи лесно хостирање на апликации во облак, за оваа цел е развиена веб апликација која овозможува корисниците да креираат, управуваат и бришат апликации на лесен начин преку кориснички интерфејс, без потреба од познавање на посложени конфигурации и концепти.

Целта на ова решение е да понуди едноставен, но моќен систем за хостирање на апликации во облак, кој ќе ја апстрахира сложеноста на инфраструктурата и ќе обезбеди интуитивен начин за градење, распоредување и управување со апликации. Главниот фокус е ставен на автоматизација, стандардизација и безбедност, со минимален број на чекори потребни од страна на корисникот. Со користење на современи cloud-native технологии, системот овозможува брзо креирање на контејнерски слики, нивно хостирање и следење, без директна интеракција со самата платформа.

Секогаш кога се бара баланс помеѓу моќност и едноставност, постои ризик од ограничување на функционалностите. Во овој случај, решението е дизајнирано да ги задоволи потребите на корисниците кои бараат брз и лесен начин за хостирање на апликации, додека истовремено обезбедува доволно можности за конфигурација и прилагодување за понапредните корисници, но изостава дел од контролата која ја овозможува платформата, и некои функционалности за контрола се оставени да се менаџираат од страна на администраторот. 

Решението нуди можности за лесно хостирање и директен пристап на нивните апликации, овозможувајќи им на корисниците да се фокусираат на развојот на нивните апликации без да се грижат за инфраструктурните детали. Апликацијата е изградена со користење на платформата Kubernetes, која овозможува автоматско управување со апликациите, скалирање и обезбедување на висока достапност, избрана беше оваа платформа поради екстензибилноста што ја нуди и начинот на кој што се менаџираат ресурсите со дадени шаблони значи дека би било полесно да се автоматизираат одредени процеси.

Градењето на апликации знае да биде комплициран процес, кој често вклучува алатки за градење, потоа потребно е овие така наречени артефакти да се хостираат на одреден сервис, и на крајот потребно е да се овозможи пристап до истите од страна на корисниците, кои подоцна од нив би се барало да ги постават на системот каде е хостирана апликацијата.

Еден предизвик со кој се соочуваат корисници е градење на своите апликации, за оваа цел најчесто се користи CI/CD (Continuous integration/ Continuous deployment) цевковод, што значи дека секој пат кога корисникот ќе направи промени во својот код, тие промени автоматски се градат и распоредуваат на платформата. Бидејќи Kubernetes платформата работи со контейнери, потребно е да се овозможи лесно градење на тие контейнери, за оваа цел беше интегриран сервис наречен kaniko кој овозможува градење на Docker слики директно во рамки на кластерот, без потреба од дополнителни привилегии. Дополнително беше потребно да се чуваат овие слики, за оваа цел беше интегриран регистар за слики кој е овозможен од страна на Docker и се нарекува Docker registry v2, кој овозможува лесно чување и повлекување на слики.

