\newpage
\section{Заклучок}

Решението претставува целосен систем за хостирање на апликации во облак, кој овозможува лесно управување и распоредување на апликации преку интуитивен кориснички интерфејс. Притоа со користење на современи технологии како Kubernetes и Docker, се обезбедува независност од инфраструктурата и се овозможува голема флексибилност на решението, доколку би било потребно, решението дозволува проширување со какви било технологии кои се потребни за одредени апликации.

Главна предност на ова решение е едноставноста за користење, при што од страна на еден корисник се бара минимален број на информации, додека при градење на слики од апликациите од корисникот потребно е само извршна датотека која има можност истиот корисник да ја изгради на свој начин, независно од решението. Ова овозможува брзо хостирање на апликација без потреба од сложени системи за градење на истите, но истовремено доколку е потребно исто така има можност за градењето слики да се прилагоди според потребите на корисниците, каде што се овозможува користење на системот за градење како CI/CD процесот (цевководот).

Апликацијата го решава проблемот што настанува при хостирање на апликации, каде што на корисниците им е потребно големо познавање од повеќе технологии, платформи, и концепти за да можат да ги хостираат своите апликации, со ова решение се овозможува лесно хостирање на апликации без потреба од познавање на сите тие концепти, додека истовремено се овозможува доволно можности за прилагодување и конфигурација за понапредните корисници.

Истовремено беше изработен интуитивен интерфејс кој овозможува лесно управување со апликации, кој што за корисниците е поделен на повеќе делови, како што се, градење на слики, организација во групи и распоредување на апликации. Интерфејсот истовремено нуди следење на апликациите во реално време, овозможувајќи на корисниците да ги следат логовите и статусот на нивните апликации во процесот на градење и распоредување. Додека за администраторите беше изработен посебен интерфејс кој овозможува контрола врз шаблоните за градење, контрола врз ресурсите на платформата, исто така со изборот на технологии се овозможува менаџирање на корисниците и нивните улоги со користење на Keycloak. Со што се минимизира потребата од интеракција со базата или со платформата со користење на корисничкиот интерфејс.

Една голема предност на решението е исто така самото хостирање, за да се хостира апликацијата (решението) не се потребни екстерни сервиси кои работат надвор од самата платформа, туку сите потребни сервиси можи да се дел од платформата, како и самата апликација што подоцна менаџира останати апликации и распоредувања. Ова исто така би овозможило лесна промена на околина.

Додека решението нуди едноставност и истовремено голема флексибилност на корисниците, исто така е присутна комплексноста која доаѓа со флексибилноста, првично апликацијата нуди поддршка за градење на Java апликации, доколку е потребна промена на технологијата за градење на слики тоа е веќе понудено од страна на апликацијата, но од страна на администраторот се бара познавање на инфраструктурата и зависностите за да има можност да креира нови шаблони за градење. 