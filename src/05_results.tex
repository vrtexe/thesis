\section{Предизвици при развој на апликацијата}

Главен предизвик при развој на оваа апликација беше да се овозможи лесно хостирање на апликации, при што корисникот не би требало да има големи познавања за работење со специфични платформи, туку едноставно да може да ја хостира својата апликација со неколку кликови. За ова решение потребно беше да се користи одредена платформа, за тоа беше избрана Kubernetes платформата, која овозможува лесно хостирање на апликации во контејнери и менаџирање на ресурсите. платформата дозволува преку дефинираниот интерфејс и потребните ресурси да се хостира апликација. За поврзување на апликацијата со Kubernetes платформата беше користена библиотеката \texttt{java-kubernetes-client} која овозможува лесно управување со ресурсите во рамки на кластерот, дава можност за интеграција со кластерот со цел да се имплементираат повеќе комплексни функционалности потребни на корисниците за преглед и за менаџирање на нивните апликации.

\subsection{Овозможување на преглед и следење низ процесот за хостирање}

Друг предизвик беше како да се овозможи преглед на распоредувањата на апликации, при тоа корисник да може да види што апликации се хостирани, за оваа цел беше искористена база на податоци за чување на сите потребни информации за градење на ресурсите кои платформата ги очекува за хостирање на апликации, информациите во базата на податоци ги следат релациите помеѓу ресурсите на платформата, и со ова се дозволува апликацијата да биде флексибилна во тоа како се хостираат апликации, при тоа корисникот има можност да ги менаџира своите апликации преку корисничкиот интерфејс, и заради тоа беше потребно да се чуваат сите информации за состојбата на ресурсите од платформата, бидејќи голема лимитација е што некои работи како Kubernetes конфигурациски мапи не се достапни веднаш до апликацијата откако се ажурира вредноста, при тоа за апликацијата да ги добие овие вредности потребно е да се направи рестарт на истата, но исто така некои вредности во ресурсите не е возможно да се ажурираат без да се избриши тој ресурс, за надминување на оваа лимитација се следи во позадина што точно ажурирал корисникот, и доколку е очекувана промена за која сме сигурни дека е возможна се праќа оваа промена до кластерот и се прави ажурирање на засегнатиот ресурс, но доколку се детектира промена која не ги задоволува ни еден од условите се прави целосно рекреирање на ресурсите, ова би значело дека корисничката апликација се рестартира.

Основна функционалност за корисници е можноста за следење на логови и статус на апликацијата во реално време, како и можноста за управување со ресурсите и можност да направи рестарт или ажурирање на сликата која се користи, за оваа цел беше потребно да се користат веб сокети кои овозможуваат двонасочна комуникација помеѓу клиентот и серверот, при што серверот во реално време испраќа информации до клиентот за статусот на апликацијата, како и логови од апликацијата, ова овозможува корисникот да има преглед на својата апликација во реално време и да може да реагира доколку нешто не е во ред со неговата апликација. За следење на овие апликации беше потребно да се креира кеш за секоја претплата за специфичните информации кои корисникот ги бара, и секој веб сокет креира своја претплата за одредени апликации, за оваа цел сите делови кои имаат отворена интеракција со кластерот за добивање на некакви информации од страна на платформата се чува информација до самиот кеш, доколку некој друг корисник ја побара истата информација не би се креирала нова врска до платформата, туку би се користела веќе постоечката врска, доколку никој не слуша на оваа врска таа се брише од кешот и се ослободуваат ресурсите.

\subsection{Градење на слики за контејнери}

За корисникот да добие можност да хостира апликација потребно беше да се овозможи на корисник да може да гради слики за контејнери, кои ги очекува Kubernetes платформата, за оваа цел најлесен начин беше да се користи алатката kaniko која овозможува градење на слики во рамки на самата платформа, без потреба од посебен Docker демон, главен предизвик тука беше како да се овозможи корисникот да може да ги прилагодува параметрите за градење на сликата, при тоа да не се ограничува само на еден начин или една технологија за градење, но исто така беше потребно да се овозможи на корисник поедноставен пристап, и не би било соодветно на корисник да се дозволи да внесува целосен Dockerfile, бидејќи тоа би барало од корисникот да има познавања за градење на слики, познавање од самата алатка која ги гради сликите. Како решение беше да се овозможи администраторот на апликацијата да креира шаблони за градење на слики, при што администраторот има можност да ги дефинира сите потребни параметри за градење на сликата, како и кои полиња ќе бидат достапни до корисникот за прилагодување, при тоа корисникот има можност да ги прилагодува само тие полиња кои се дефинирани од страна на администраторот, со што се овозможува флексибилност во градењето на сликите, но исто така се олеснува процесот за корисникот кој не би требало да има големи познавања за градење на слики. Голема придобивка беше тоа што корисникот добива само неколку полиња за прилагодување, и како едноставна опција им е понудено да прикачат извршна датотека, од која се креира сликата, при тоа за конфигурација се праќаат потребни параметри.

Градењето на слики за контејнери беше имплементирано со прикачување на извршна датотека од страна на корисникот, но за истата да се прати до WebDAV серверот беше потребно да се додади поддршка за прикачување на датотеки но истовремено и да се пратат потребните параметри за градење на сликата, но доколку не беше потребно да се праќа извршна датотека, туку само праќање на параметри, за ова беше потребно да се овозможи и таква опција. Затоа овие два дела остануваат посебни во апликацијата, и е дефиниран посебен повик доколку се праќа извршна датотека, и посебен повик во случајот каде не се испраќа датотеката, туку само параметри.

\subsection{Простор за чување на извршни датотеки и слики}

Друг проблем беше каде да се чуваат овие извршни датотеки, бидејќи овие датотеки е потребно да се пристапат во рамки на кластерот од страна на алатката за градење на слики, но не беше возможно да се праќаат директно до алатката за градење, за оваа цел беше искористен WebDAV сервер кој е хостиран исто така во рамки на платформата, овој сервер е скриен за корисници, и тие немаат директна интеракција со истиот, интеракцијата се врши преку апликацијата, за комуникација со WebDAV серверот се користи WebDAV клиент за Java наречен \texttt{sardine}, овој клиент целосно го имплементира WebDAV протоколот.

Изградените слики се чуваат во регистар за слики, при што беше потребно да се пристапат овие слики во рамки на кластерот, за оваа цел беше искористен Docker registry кој е хостиран во рамки на платформата, овој регистар овозможува чување на слики и пристап до истите од страна на кластерот, за да се овозможи чистење на сликите беше потребно да се комуницира со регистарот преку неговиот интерфејс, за ова се користи библиотеката \texttt{spring-webflux} која овозможува лесно праќање на HTTP барања и примање на одговори од регистарот.

\subsection{Развој на апликацијата и локално тестирање}

Најголем предизвик беше развојот на апликацијата бидејќи апликацијата е длабоко интегрирана со Kubernetes платформата, затоа за тестирање на апликацијата беше потребно да се има пристап до кластер, за оваа цел беше искористен kind кој овозможува креирање на локален кластер, но тоа не беше доволно, исто така за полесен развој апликацијата содржи две можности за започнување на истата, првата можност беше да се користи од внатрешноста на кластерот, за оваа цел постои Spring профил наречен \texttt{cluster} во кој се конфигурираат сервисите преку нивните внатрешни адреси, при што апликацијата се стартира како дел од кластерот и се користат сите сервиси кои се хостирани во рамки на кластерот. Втората можност беше да се користи Spring профил наречен \texttt{external}, при што сите сервиси се пристапуваат преку нивните надворешни адреси, ова овозможува развој на апликацијата од локална машина, при што апликацијата комуницира со сервисите кои се хостирани во рамки на кластерот и со самиот кластер, но самата апликација не е дел од кластерот.

За да се овозможи комуникација од надвор од кластерот се користи Nginx ingress контролер кој овозможува пристап до сервисите од надворешната мрежа, и потребно беше истиот да се интегрира во рамки на кластерот. Дополнително потребно беше да се интегрираат сите дополнителни сервиси потребни за развој на апликацијата, како што се базата на податоци, WebDAV серверот, keycloak серверот, и регистарот за слики, сите овие сервиси беа интегрирани во рамки на кластерот за да се овозможи целосно тестирање на апликацијата во реални услови. За интеграција со кластерот се користи апликацискиот интерфејс на кластерот кога апликацијата се наоѓа во рамки на кластерот потребно беше да се конфигурираат улогите кои апликацијата ги добива, за оваа цел беше искористен RBAC системот на Kubernetes кој овозможува дефинирање на улоги и дозволи за пристап до ресурсите во рамки на кластерот.

За тестирање на градењето на сликите се користеше kubectl\cite{kubectl} алатка која овозможува интеракција со кластерот, при тоа за менаџирање на потребните ресурси за локално тестирање, доколку една алатка не е соодветно наместена апликацијата не може да функционира правилно, ова е голем проблем бидејќи апликацијата има многу екстерни зависности, за оваа цел беше направена скрипта која овозможува лесно поставување на сите потребни сервиси во рамки на кластерот, при тоа се овозможува брзо поставување на целата околина за тестирање на апликацијата.

Во случај како што се WebDAV или регистарот за слики, се користеше алатка како curl за тестирање на овие сервиси\cite{webdav_usage}.

\subsection{Автентикација и авторизација}

Автентикацијата и авторизацијата беше имплементирана со помош на Keycloak серверот, кој овозможува лесно управување со корисници и улоги, за оваа цел беше потребно да се интегрира Keycloak серверот во рамки на кластерот, и да се конфигурираат сите потребни клиенти и улоги за апликацијата. Главен предизвик беше како да се овозможи контрола на пристап на различни ресурси во апликацијата за оваа цел беше искористен OpenID Connect протоколот кој овозможува авторизација и автентикација помеѓу корисникот и серверот. Дополнително беше потребно да се овозможи интеграција на Keycloak серверот со самата апликација, за оваа цел беше искористен Spring Security кој овозможува лесна интеграција со OAuth2 протоколот и Keycloak серверот, Додека на корисничкиот интерфејс беше искористена библиотеката што ја развиваат луѓето што работат на Keycloak наречена \texttt{keycloak-js}\cite{keycloak_js} која овозможува лесна интеграција на Keycloak серверот со веб апликации.

\section{Подобрувања и осврт кон иднината}

Апликацијата беше развиена со основните функционалности во предвид, при тоа главен фокус беше да се овозможи лесно хостирање на апликации во рамки на Kubernetes платформа, но постојат неколку подобрувања кои би можеле да се направат во иднина за да се подобри корисничкото искуство и функционалностите на апликацијата, едно такво подобрување би било да се овозможи поддршка за сервиси како што се Kafka, RabbitMQ, бази на податоци, додека во моментот ова е можно да се изведи со користење на апликацискиот интерфејс наменет за кориснички апликации, но посветено решение за ова би овозможило поголема контрола, и полесно управување од страна на администраторите и корисниците.

Дополнително би можело да се овозможи поддршка за повеќе платформи за хостирање на апликации, како што се Docker Swarm, OpenShift, при што администратор би имал можност да избере која платформа сака да ја користи за хостирање и да го вметни решението на истата платформа. Исто така една лимитација е тоа што градењето на слики е врзано со самата платформа, едно подобрување би било апликацијата да овозможи поддршка за прикачување на слики до користениот регистар, при што корисникот би имал можност да гради слики на свој начин и да ги прикачува до регистарот, наместо да се користи вградената функционалност за градење на слики.

Друга можност за подобрување би било да се овозможи контрола врз страниците со документација, каде што на администратор би се дозволило да управува со содржината, со ова на корисниците би се овозможило подобро искуство при користење, затоа што администраторот ги познава своите корисници и знае повеќе што би им било потребно.
